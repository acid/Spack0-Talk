% !TEX encoding = UTF-8 Unicode
\documentclass{beamer}

\usepackage{beamerthemesplit} 
\usepackage[utf8]{inputenc}


\title{Der Kampf für die Informationsfreiheit ist ein sozialer}
\author{Daniel Schweighöfer}
\date{\today}

\begin{document}

\frame{\titlepage}

\section[Einführung]{}
\frame{\tableofcontents}

\section{Informationsfreiheit im und durch das Netz}
\subsection{Echt ne gute Sache...}
\frame
{
  \frametitle{denn sie...}

  \begin{itemize}[<+->]
  \item fördert die Entwicklung von Politik und Kultur
  \item lässt Menschen näher zusammen wachsen
  \item verstärkt soziale Phänomene
  \end{itemize}
}
\subsubsection{Beispiele für positive soziale Phänomene}
\frame
{
  \frametitle{Hypes}
    Pic to come
}
\frame
{
  \frametitle{Graswurzel-Bewegungen}
    Another pic to come
}
\frame
{
  \frametitle{Flow of Art}
    Jet another pic to come
}
\subsection{... aber auch mit großem Haken}
\frame
{
  \frametitle{denn sie...}

  \begin{itemize}[<+->]
  \item fördert das Aufzeigen von kulturell-/sozialen Unterschieden
  \item erleichtern soziale Isolation/Seperation
  \item verstärkt soziale Phänomene
  \end{itemize}
}
\subsubsection{Beispiele für negative soziale Phänomene}
\frame
{
  \frametitle{Shitstorms}
  pic2come
}
\frame
{
  \frametitle{Raids}
  pic2come
}
\section{It's all about privileges!}
\subsection{Das Märchen vom Default-Menschen}
\frame
{
  \frametitle{Sei...}
  
  \begin{itemize}[<+->]
  \item weiß,
  \item männlich,
  \item sowie nicht zu alt
  \end{itemize}
  
  \begin{block}<4->{Und die Welt gehört dir!}
  \end{block}
}
\frame
{
  \frametitle{Exkurs: Konzept der einen Wahrheit}
  
  Mensch geht davon aus, dass es eine allgemeingültige Wahrheit gäbe.
  Davon abweichende Ansichten werden
  \begin{itemize}
    \item grundsätzlich in Frage gestellt
    \item als weniger gültig betrachtet.
  \end{itemize}
}
\subsection{Privilegien}
\frame
{
  \frametitle{Wer?}
  
  Menschen, denen die arbiträren "defaults" zugesprochen werden, sind privilegiert.
}
\frame
{
  \frametitle{Wie?}
  
  Priviligiert sein heißt
  \begin{itemize}[<+->]
  \item von anderen bevorzugt bzw. nicht besonders behandelt werden
  \item sich sozialem Rückhaltes sicher sein
  \end{itemize}
}
\frame
{ 
  \frametitle{Was ist daran schlecht?}
  
  Menschen in der Informationsfreiheit brauchen
  \begin{itemize}[<+->]
  \item soziale Sicherheit
  \item (maximal) erweiterte Schutzräume 
  \item freie Wahl von Identitäten
  \end{itemize}
  
  \begin{theorem}<4->
    Eine von Privilegien geprägte Gesellschaft wie die unsere jetzige droht die positiven Effekte der Informationsfreiheit zu negieren! 
  \end{theorem}
}
\subsection{Mechanismen zur Etablierung und Stärkung von Privilegien}
\frame
{
  \frametitle{Ein bunter Strauß Begriffe aus den Sozialwissenschaften}
  
  \begin{itemize}[<+->]
  \item Exklusion
  \item Fremdmarkierung
  \item Normen
  \item Geld
  \item systematischer Matcherhalt
  \end{itemize}
}
\section{Aktionsvektoren}
\subsection{-ismen ftw!}
\frame
{
  \frametitle{Pluralismus}
  
  Die Diversität der Lebens- und Denkweisen muss gefördert werden!
}
\frame
{
  \frametitle{Feminismus}
  
  Eine wichtige Bewegung die den meisten kaum vertraut ist. Das sollten wir ändern, um
  \begin{itemize}[<+->]
  \item voneinander zu lernen und
  \item einen eigenen, offeneren Feminismus zu prägen.
  \end{itemize}

}
\frame
{
  \frametitle{Anarchiismus}
  
  Viele bekannte Probleme aber auch verstaubte Geister.
  \begin{itemize}[<+->]
  \item Menschenbildproblematik
  \item Diskussionskultur
  \end{itemize}

}
\frame
{
  \frametitle{Erklärbär(-ismus)}
  
  \begin{itemize}[<+->]
  \item unheimlich wichtig
  \item unheimlich anstrengend
  \item unheimlich belohnend
  \end{itemize}

}
\subsection{Die große Frage}
\frame
{
  \frametitle{Wie auf einen Angriff reagieren?}
  
  \begin{itemize}[<+->]
  \item wehren =&gt; anpassen => nicht weiter entwickelt
  \item fliehen => aufgeben => aus der Evolution verabschieden
  \item reden? energisch friedlich bleiben? Auf den großen Crash warten?
  \end{itemize}

}
\frame
{
  \frametitle{Danke!}
  
  \begin{itemize}
  \item Fragen?
  \item Kritik?
  \item Anregungen?
  \end{itemize}

}
\end{document}
